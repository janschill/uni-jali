\documentclass[runningheads]{llncs}
\usepackage[T1]{fontenc}
\usepackage{listings}
\usepackage{graphicx}
\usepackage[inline]{enumitem}
\usepackage{hyperref}
\usepackage[acronym]{glossaries}
\makeglossaries
\newacronym{html}{HTML}{Hypertext Markup Language}
\newacronym{adt}{ADT}{algebraic data type}
\newacronym{dom}{DOM}{Document Object Model}
\newacronym{vdom}{VDOM}{Virtual Document Object Model}
\newacronym{mvu}{MVU}{Model-View-Update}
\newacronym{css}{CSS}{Cascading Style Sheets}
\newacronym{ast}{AST}{abstract syntax tree}
\usepackage{float}
\usepackage{upquote}
\usepackage{fancyvrb}
\usepackage{color}
\definecolor{bluekeywords}{rgb}{0.13,0.13,1}
\definecolor{commentscolor}{rgb}{0.5,0.5,0.5}
\definecolor{greenstrings}{rgb}{0,0.5,0}
\definecolor{codepurple}{rgb}{0.58,0,0.82}
\definecolor{codered}{rgb}{0.8, 0.25, 0.33}
\definecolor{codeorange}{rgb}{1.0, 0.49, 0.0}
\definecolor{codebrown}{rgb}{0.8, 0.58, 0.46}

\lstset{escapeinside={(*@}{@*)}}

\lstdefinelanguage{FSharp}%
{ morekeywords={[0]{let, new, match, with, rec, open, module, namespace, type, of, member, and, for, while, true, false, in, do, begin, end, fun, function, return, yield, try, mutable, if, then, else, cloud, async, static, use, abstract, interface, inherit, finally }},
  otherkeywords={ let!, return!, do!, yield!, use!, var, from, select, where, order, by },
  keywordstyle={[0]{\color{codeorange}}},
  morekeywords={[1]{Value, int, bool, char, string, list, Expr, Env, ADTConstructor}},
  keywordstyle={[1]{\color{codered}}},
  sensitive=true,
  basicstyle=\ttfamily,
  breaklines=true,
  xleftmargin=\parindent,
  aboveskip=\bigskipamount,
  tabsize=2,
  morecomment=[l][\color{greencomments}]{///},
  morecomment=[l][\color{greencomments}]{//},
  morecomment=[s][\color{greencomments}]{{(*}{*)}},
  morestring=[b]",
  showstringspaces=false,
  literate={`}{\`}1,
  stringstyle=\color{redstrings},
  numbers=left,
  stepnumber=1,
}
\definecolor{bluekeywords}{rgb}{0.13,0.13,1}
\definecolor{greencomments}{rgb}{0,0.5,0}

\lstdefinelanguage{JaLi}%
{ keywords={[0]{let, match, with, type, true, false, end, func, function, if, then, else}},
  keywordstyle={[0]\color{codeorange}},
  morekeywords={[1]{Node, String, Tuple, Msg, List, Integer}},
  keywordstyle={[1]{\color{codered}\bfseries}},
  sensitive=true,
  numbers=left,
  stepnumber=1,
  basicstyle=\ttfamily,
  breaklines=true,
  xleftmargin=\parindent,
  aboveskip=\bigskipamount,
  tabsize=2,
  morecomment=[l][\color{commentscolor}]{///},
  morecomment=[l][\color{commentscolor}]{//},
  morecomment=[s][\color{commentscolor}]{{(*}{*)}},
  morestring=[b]',
  showstringspaces=false,
  literate={`}{\`}1,
  stringstyle=\color{greenstrings},
}

\lstdefinelanguage{Other}%
{ sensitive=true,
  numbers=left,
  stepnumber=1,
  basicstyle=\ttfamily,
  breaklines=true,
  xleftmargin=\parindent,
  aboveskip=\bigskipamount,
  tabsize=2,
  morecomment=[l][\color{commentscolor}]{///},
  morecomment=[l][\color{commentscolor}]{//},
  morecomment=[s][\color{commentscolor}]{{(*}{*)}},
  morestring=[b]",
  showstringspaces=false,
  literate={`}{\`}1,
  stringstyle=\color{greenstrings},
}
\begin{document}

\title{%  
  Optimization of the model-view-update pattern
  by using partial evaluation\break%
}

\author{Liv Hartoft Borre 
\and Jan Schill
}

\authorrunning{L. H. Borre \and J. Schill}

\institute{
IT University of Copenhagen, Copenhagen, Denmark\\
\email{\{livb,schi\}@itu.dk}
}

\maketitle

\begin{abstract}
    Modern techniques to perform DOM updates on rendered HTML documents mostly use a VDOM with comparison of two views: the current view and an updated one after an action is performed. The comparison of tree-structures with subsequent patching of the structure and rerendering is computational expensive. With clever compiler optimizations using partial evaluation these comparisons can be reduced. An action should always make changes in a predictable manner and in the same position in the DOM. With partial evaluation the expensive comparison can be precomputed at compile time and reduced to a single direct change that circumvents the update-model-view cycle.
This paper will cover this compile time optimization and the generation of JavaScript snippets, with which the change can be performed and demonstrate that the VDOM comparison can be prevented to a certain degree. The paper will not introduce the compiler to JavaScript that would generate a complete working program for the browser.

\end{abstract}

\newcommand{\jan}[1]{\par\smallskip\noindent{\small\llap{\textbf{{jan:}~~}}{\textsf{#1}}}\par\smallskip}
\newcommand{\liv}[1]{\par\smallskip\noindent{\small\llap{\textsf{{liv:}~~}}{\textsl{#1}}}\par\smallskip}

\section{Introduction}
Many ideas have been shaped in order to make websites more dynamic and avoid breaking the site when loading a new \gls{html} file from the server. One pattern that solves this problem efficiently is the \gls{mvu} application pattern.
To solve the problem of showing diverged information in an \gls{html} document, the JavaScript language is needed, as there is no native and complex enough support for \gls{html} or \gls{css} to force the browser to show this new information. JavaScript on the other hand can directly inject into the \gls{dom}, by having complete access to every \gls{html} node element in the rendered \gls{dom}.
JavaScript can select a node element by \texttt{ID} and change its displayed value, forcing the browser to rerender the \gls{dom} on that node. Figure \ref{fig:button_browser} shows a minimal example, where JavaScript is used to alter the rendered \gls{html}.

\begin{figure}
    \centering
    \includegraphics[width=0.8\textwidth]{images/button-browser.png}
    \caption{Simple button counter example}
    \label{fig:button_browser}
\end{figure}

This way of updating content on the document directly has been evolved over the years and is heavily used to build any sized modern websites. With the help of frameworks like React\footnote{\url{https://reactjs.org/}}, a JavaScript framework developed by Facebook or Elm\footnote{\url{https://elm-lang.org/}}, a framework utilizing its own functional language, it has become easier to go from a small button example to a fully working componentized website.
The Elm framework uses the \gls{mvu} application pattern to realize the idea of dynamically updating the \gls{dom}.

The basic functionality how Elm uses it, is as follows. The framework generates \gls{html} that can be viewed by a client. This client can then interact with the \gls{html} by clicking a button for example. This produces a \texttt{Msg (Message)}, which is send to Elm, that generates another \gls{html} document. This cycle is what is referred to as the \gls{mvu} cycle.
In the Elm program three parts play a crucial role: The Model, View and Update.

When a message of type \texttt{Msg} is produced, the update function is being called, which takes the message and the current state of the model and returns an updated model. The view function, which is responsible of generating the \gls{html} that can be viewed by the browser, is then called automatically, rendering the \gls{html} with the updated model.

\begin{lstlisting}[columns=fullflexible, label={view_model}, language=Other, caption={View, update and model and their types}]
view : Model -> Html Msg
update : Msg -> Model -> Model
\end{lstlisting}

\paragraph{How does Elm optimize the view rendering?}
Oftentimes when a new view is generated and needs to be re-rendered, it is not necessary for the whole document to be switched out, but rather only the \gls{dom} node that experienced change. Forcing the browser to generate new \gls{dom} nodes is computational expensive work.
For this most frameworks use a \gls{vdom}, which resembles the complete active \gls{dom} in memory in a convenient data structure. The \gls{vdom} is generated with the updated model and compared against the current active \gls{vdom}, finding the parts that need updating. This part is called \textit{diffing} and it is done by iterating over both representations of the \gls{dom} and comparing each individual node. This \textit{diffing} then generates a data structure that keeps track of all changes. A \textit{patch} function takes the old view and the data structure holding the changes, which then updates all occurrences of change in the view.

All the \textit{diffing} and \textit{patching} against the \gls{dom} tree seems to be still computational expensive, as in the end, only the following code snippet is actually needed for the button example to work:

\begin{lstlisting}[columns=fullflexible, label={button_increment}, language=Other, caption={Reduced example to show only the incrementing}]
<html>
  <button onclick=increment()>Increment</button>
  <p id="model">0</p>
</html>
<script>
  let count = 0;
  const model = document.getElementById('model');
  function increment() {
    model.textContent = ++count;
  }
</script>
\end{lstlisting}

\paragraph{How does JaLi optimize the view rendering even further?} This paper will outline all the needed parts to use partial evaluation, symbolic execution and constant folding to optimize the view rendering. It will be shown on the small button example how the compiler of the JaLi language can reduce all the previously mentioned parts that are computational expensive, by applying the optimizations.
Due to time restrictions and the sheer complexity of the project it was not able to implement the full working compiler optimizations, but enough to demonstrate reduction on certain programs. A compiler that transforms a full program to a \gls{html} and JavaScript website that can run in the browser is also not implemented---this will be simulated by manual coding the JavaScript parts. Nevertheless, the theory has been established and it will be shown on small examples to illustrate the idea.

In the following chapter the functional language that was developed for this project will be explained. The next chapter describes partial evaluation and how reduction is carried out on the JaLi language. Next the implementation of the \gls{mvu} in the JaLi language is shown, and lastly the button example will be used to show how it can reduced into functioning \gls{html} and JavaScript that avoids the \gls{mvu} cycle and the \gls{vdom} diffing.

\newpage

\section{The JaLi language} \label{jali}
The JaLi language is a minimal higher-order functional language without type checking. The syntax is closely coupled to FSharp and Haskell.

\lstinputlisting[label={jali_button_example}, language=JaLi, caption=Button example written in JaLi]{./code/button.jali}
    
Listing \ref{jali_button_example} shows how the button example from the previous chapter could look like in the JaLi language.
It introduces an \gls{adt} called \texttt{Node} in line 1, that has two constructors: \texttt{Text} and \texttt{Tag}. An \gls{adt} is a composition type, which means that it is used to introduce new and more complex types to a language. This is especially useful when wanting to depict a type from the outside into the language. This is illustrated by the \texttt{Node} type in the given example. The \texttt{Node} type is representing an \gls{html} document element. In \gls{html} an element always carries an element name, for example \texttt{button}: \texttt{<button></button>}. It has a list of key-value pairs, named as \texttt{attributes}: \texttt{id=buttonIncrement} , which are set after the element name in the opening tag. Lastly, it holds elements or text in between its opening and closing tags.

\begin{lstlisting}[columns=fullflexible, label={button_increment_html}, language=JaLi, caption=Button increment in HTML]
<button id=buttonIncrement>Increment</button>
\end{lstlisting}

Represented in JaLi as an \gls{adt} the single button \gls{html} element would look like in listing \ref{button_increment_jali}.

\begin{lstlisting}[columns=fullflexible, label={button_increment_jali}, language=JaLi, caption=Button increment in JaLi as ADT]
Tag ('button') ([('id', 'buttonIncrement')]) ([
  Text 'Increment'
])
\end{lstlisting}

Just like in \gls{html} then, the \gls{adt} can recursively be nested to construct a tree-like structure.
Another example for this would be a recursive list data type

\begin{lstlisting}[columns=fullflexible, label={recursive_list}, language=JaLi, caption=Recursive list data type]
type List = Nil | Cons Integer List;

Cons 1 (Cons 2 (Cons 3 (Nil)))
\end{lstlisting}

An \gls{adt} always carries super-type defined right after the keyword \texttt{type}, followed by a number of constructors, which are used to instantiate an actual value of the defined \gls{adt}. These constructors can carry optional type parameters, which need to be given, when creating it. Like in listings \ref{button_increment_html} and \ref{button_increment_jali} the \texttt{Tag} constructor needs a string and two lists. It should be noted that the JaLi language it untyped, and that these types carry nothing but descriptive values as well as define the arity of the constructor, i.e. the number of arguments it requires. 

\glspl{adt} make mostly only sense with pattern matching implemented as well. Pattern matching will allow to control the flow of the program according to the state of the matching \gls{adt} values.

\begin{lstlisting}[columns=fullflexible, label={pattern_match}, language=JaLi, caption=Pattern matching on Msg]
type Msg = Increment | Decrement;
message = Increment;

match message with
  | Increment -> 'Increment the model'
  | Decrement -> 'Decrement the model'
  | _ -> 'Unknown action'
\end{lstlisting}

The example in listing \ref{pattern_match} shows the known \gls{adt} definition in line 1. After it on line 2 a \textit{let binding} can be seen, where the value on the right side is assigned to the variable called \texttt{message}. Afterwards this binding is used in pattern matching context. The value of \texttt{message} will be matched with the available patterns defined after the pipe symbol. There are three patterns that can be matched with, each valid action and one \textit{wild card} pattern, that acts like the \textit{else} block in an \textit{if statement}. The complete \texttt{match} block will then return only the right side of the matched pattern.

Functions are defined by enclosing a name, parameters and a function body between the key words \texttt{func} and \texttt{end}.

\begin{lstlisting}[columns=fullflexible, label={inner_functions}, language=JaLi, caption={Inner functions and recursion shown on factorial}]
func factorial n =
  func mult x y =
    x * y
  end
  
  if n == 0
  then 1
  else mult (n) (factorial (n - 1))
end
factorial 5 // => 120
\end{lstlisting}

Functions can be recursively called and can even have inner functions.

\subsection{Grammar}\label{Grammar}
The lexer, parser and interpreter are all written in FSharp using the libraries \textit{FsLex} and \textit{FsYacc}.

\subsubsection{Lexer}
The lexer receives a JaLi program as a string, recognizes and transforms its characters in the program to tokens in FSharp. These tokens are defined in the \textit{Lexer.fsl} and \textit{Parser.fsy}. The lexer does this by having a defined list of regular expressions to tokens and matches the input string character by character to those patterns.
The result is a list of tokens that might end up looking lik listing \ref{lexer_input}

\begin{lstlisting}[columns=fullflexible, label={lexer_input}, language=JaLi, caption=Lexer result]
// JaLi program as string:
"5 + 10"
// Result of lexing the string:
Parser.token = CONSTINT 5
Parser.token = PLUS
Parser.token = CONSTINT 10
Parser.token = EOF
\end{lstlisting}

After the whole string is successfully transformed into tokens, the parser will try to make sense of it.

\subsubsection{Parser}
The parser receives the list of tokens after the lexical analysis and builds an \gls{ast} from it. It does this by recognizing different combinations of tokens, that are defined explicitly as showcased by listing \ref{parser_expression}.

In the case of JaLi, a program---indicated by is point of entry with \texttt{Main}---is an \texttt{Expression}, which is defined in the parser as the following:

\begin{lstlisting}[columns=fullflexible, label={parser_expression}, language=FSharp, caption=Program defined by \texttt{Expression}]
Main:
    Expression EOF { $1 }

Expression:
  | AtomicExpression { $1 }
  | FunctionCall { $1 }
  | ConditionalExpression { $1 }
  | Baseoperation { $1 }
  | Binding { $1 }
\end{lstlisting}

From this, it is not obvious how a program then can be multiple \texttt{Bindings} or \texttt{Functions} with a \texttt{FunctionCall} in the end. This is realized by defining a \texttt{Binding} by its \texttt{Expression} that it binds, but also expecting another \texttt{Expression} right after the binding.

\begin{lstlisting}[columns=fullflexible, label={parser_binding}, language=FSharp, caption=Binding with \texttt{Expression} afterwards]
Binding:
  | LocalBinding { $1 }
  | FunctionBinding { $1 }
  | ADTBinding { $1 }
  
LocalBinding:
  | NAME ASSIGN Expression SEMICOLON Expression { Let($1, $3, $5) }
\end{lstlisting}

The \texttt{SEMICOLON} marks an end of a \texttt{Binding}, indicating that the \texttt{Expression} after it, is the next \texttt{Expression} in the program---allowing multi-line programs and not just a single line program. The part in curly braces after a parser rule was up until here completely ignored but shall be explained as being the mapping to its \gls{ast} element. A \texttt{Binding} in the \gls{ast} is represented by a \texttt{Let} key word, which holds three values, these are passed over by the call on line 7 in listing \ref{parser_binding}.
The \texttt{Let} is specified as \texttt{Let of string * Expr * Expr} in the \gls{ast} file. This means that the first parameter needs to be of type string, the name and the other two parameters of type \texttt{Expr}, which is a defined type.

\begin{lstlisting}[columns=fullflexible, label={ast}, language=FSharp, caption=An excerpt of the \gls{ast} of JaLi]
type Value =    
    | IntegerValue of int
    | BooleanValue of bool
    | CharValue of char
    | StringValue of string
    | TupleValue of Value * Value
    | ListValue of Value list
    | ADTValue of string * string * Value list
    | Closure of string * string list * Expr * Value Env
    | ADTClosure of ADTConstructor * string * Value Env

and Expr =
    | ConcatC of Expr * Expr
    | List of Expr list
    | Constant of Value
    | StringLiteral of string
    | Variable of string
    | Tuple of Expr * Expr
    | Prim of string * Expr * Expr
    | Let of string * Expr * Expr
    | If of Expr * Expr * Expr
    | Function of string * string list * Expr * Expr
    | ADT of string * ADTConstructor list * Expr
    | Apply of Expr * Expr list
    | Pattern of Expr * (Expr * Expr) list
\end{lstlisting}

\vspace{0.3cm}

The listing \ref{ast} gives a rough overview of the multiple \texttt{Expressions} that are currently implemented and reveal a bit of its possibilities.

Returning back to the grammar in the parser the implementation and characteristic of the JaLi grammar shall be explained.
As already mentioned, everything reduces to an \texttt{Expression}, which are split into the ones shown in listing \ref{parser_expression}.

We distinguish between \texttt{Atomic Expression} and other expressions. An \texttt{Atomic Expression} is a constant, a variable, a tuple, a list, a cons operator, or a parenthesized expression (so that it is easy to see where an \texttt{Atomic Expression} begins and ends). Constants are integers, booleans, strings, and the wildcard character '\_' which is used for match expressions.

The reason we make this distinction is because syntax like function applications without parentheses makes the language ambiguous, as we then cannot decide when the function application expression ends, and another begins. A solution to this, is to distinguish between \texttt{Atomic Expression} and other expressions, and then require arguments for function application to be atomic. 
\\\\
Besides \texttt{Atomic Expression}, an expression can be a logical negation, a base operation for arithmetic and comparison (+, -, ==), a conditional expressions, a match expressions, a function call or a binding. A binding is an \gls{adt} declaration, a local binding, or a function declaration. Match expressions consist of a match expression and a list of patterns. Each pattern is a tuple of expressions, where the first expression is the pattern to match against the match expression, and the second is the body to execute if the pattern matches. The \gls{adt} consist of a name and a list of constructors. Each of these constructors again contain a name and the list of type parameters. The types describe the arity of the constructor, and provides descriptive value, but it does not imply any actual type constraints on its arguments. 

\begin{Verbatim}[numbers=left,xleftmargin=\parindent]
Main ::=
  expr

expr ::=
  atomicexpr                            atomic expression
  !expr                                 logical negation
  expr op expr                          base operations, arithmetic, 
                                        comparison
  if expr then expr else expr           conditional expression
  match expr with patterns              match expression
  NAME atomicexprs                      function call
  binding                               types, locals, functions

atomicexpr ::=
  const                                 constant literals
  NAME                                  local variable or parameter
  (expr, expr)                          tuple
  [ expr, expr, ..., expr ]             list
  expr::expr                            cons operator
  ( expr )                              parenthesized expression

const ::=
  CONSTINT                              integer literal
  CONSTBOOL                             boolean literal
  STRINGLITERAL                         string literal
  USCORE                                wildcard literal

binding ::=
  type NAME = constructors; expr        abstract data type binding
  NAME = expr; expr                     local binding
  function NAME params = expr end expr  function binding

patterns ::=
  | expr -> expr                        one pattern
  | expr -> expr patterns               more than one pattern

constructors ::=
  NAME typeparams                       one type constructor
  NAME typeparams | constructors        more than one type constructor

type ::=
  INT                                   Integer type
  FLOAT                                 Float type
  BOOLEAN                               Boolean type
  STRING                                String type
  CHAR                                  Char type
  NAME                                  Name variable type
  ( type, type )                        Tuple of types
  [ type ]                              List of type
  
typeparams ::=
  (* empty *)                           zero type parameters
  type typeparams                       more than zero type parameters

atomicexprs ::=
  atomicexpr                            one expression
  atomicexpr atomicexprs                more than one expression

params ::=
  NAME                                  one parameter
  NAME params
\end{Verbatim}


\subsubsection{Interpreter}
An interpreter takes an expression, evaluates it and returns a value. The interpreter implemented for JaLi is also written in FSharp.
When given an expression from the list defined in the \gls{ast}, it matches this expression with the correct pattern and returns its value.

\begin{lstlisting}[columns=fullflexible, label={interpreter-function_head}, language=FSharp, caption={Function head of \texttt{eval} function in interpreter}]
let rec eval (e: Expr) (env: Value Env): Value =
    match e with
    // ...
    | And (expression1, expression2) ->
        match eval expression1 env with
        | BooleanValue false -> BooleanValue false
        | BooleanValue true -> eval expression2 env
        | _ -> failwith Can only use boolean values
    // ...
\end{lstlisting}

Listing \ref{interpreter-function_head} shows the head of the \texttt{eval} function. It takes two arguments: The expression and an environment variable of type \texttt{Env}, which is a list of tuples of type string and the type parameter provided.
This environment is used to store variables and functions. This will be explained further on a concise example.
The pattern matching from listing \ref{interpreter-function_head} gives the evaluation of an \texttt{And} expression, which is the logical conjunction.
It operates on two expressions, but because an expression is the base case for the JaLi language, those expressions can of course only be evaluated to a boolean value, if they are boolean values themselves. Therefore, both expressions need to be evaluated to the type \texttt{BooleanValue} before they can be operated on. In the logical conjunction it is only necessary to evaluate the second expression, when the first evaluates to true, because if the first expression is false, it does not matter what the second expression holds, as it will never evaluate to true.

The interpreter also needs to account for the problem of wanting to have more than just one single expression in a program. Those were handled for example at \textit{let bindings}, which consists of two expressions, the one it binds to and all expressions after it.

\begin{lstlisting}[columns=fullflexible, label={interpreter-let_binding}, language=FSharp, caption={Evaluation of let bindings}]
    // ...
    | Let (name, expression1, expression2) ->
        let value = eval expression1 env
        let newEnv = (name, value) :: env
        eval expression2 newEnv
    // ...
\end{lstlisting}

Because a \textit{let binding} is essentially not a full program by itself, as it only binds an evaluated expression to a variable and not returning it by itself until it is explicitly called, it needs to do three operations:

\begin{enumerate}
    \item It needs to evaluate the expression that it is supposed to bind to.
    \item It needs to use the provided environment, to make this variable and its expression available for expressions following it.
    \item It needs to continue the interpretation of the program by evaluating the second expression, while providing the new environment
\end{enumerate}

One interesting aspect about evaluating programs are the handling of functions. Function declarations will be---as previously explained---put into the environment, before this happens though, they are wrapped in a type \texttt{Closure}, which holds the function name, the parameters it expects, the current environment and the expression, which is the function body.

\begin{lstlisting}[columns=fullflexible, label={interpreter-function_binding}, language=FSharp, caption={Evaluation of function bindings}]
    // ...
    | Function (name, parameters, expression, expression2) ->
        let closure =
            Closure(name, parameters, expression, env)
        let newEnv = (name, closure) :: env
        eval expression2 newEnv
    // ...
\end{lstlisting}

When the \gls{ast} type \texttt{Apply}---the call of a function---is encountered, it first looks up the \texttt{Closure} in the environment, then makes sure that not too many arguments are provided, if less arguments are given then the function is defined with, it will return another \texttt{Closure}, partially applying the function.
This will be important for partial evaluation in chapter \ref{reducer} to make the optimizations that are planned.
In both cases of less arguments and equal arguments to the parameters, the interpreter will bind those values, to the parameters, making them available for the function body to apply.

\begin{lstlisting}[columns=fullflexible, label={interpreter-function_param_binding}, language=FSharp, caption={Binding of arguments and parameters in a function call}]
let newEnv =
    List.fold2 (fun dEnv parameterName argument ->
      (parameterName, eval argument env) :: dEnv)
      ((cname, fclosure) :: declarationEnv) cparameters farguments
\end{lstlisting}

Another important and interesting case is the pattern matching evaluation. For this a \texttt{Pattern} type with the \texttt{matchExpression} and a list of patterns is mapped on.
Firstly, the \texttt{matchExpression}, which is the expression that is desired to find in the patterns, is evaluated. The value from this operation is then being matched against the different patterns given and in general available.

\begin{lstlisting}[columns=fullflexible, label={interpreter-pattern}, language=FSharp, caption={Evaluation of pattern matching}]
    // ...
    | Pattern (matchExpression, (patternList)) ->
        let matchPattern x (case, expr) =
            tryMatch (tryLookup env) x case
            |> Option.map (fun bs -> (case, expr, bs))

        let evaluatedMatch = eval matchExpression env
        match List.tryPick (matchPattern evaluatedMatch) patternList with
        | Some (case, expr, bindings) -> env @ bindings |> eval expr
        | None -> failwith Pattern match incomplete
    // ...
\end{lstlisting}

\begin{lstlisting}[columns=fullflexible, label={interpreter-pattern_match}, language=FSharp, caption={Helper function to find pattern and bind values}]
let rec tryMatch (lookupValue: string -> option<Value>)
                   (actual: Value) (pattern: Expr) =
    let tryMatch = tryMatch lookupValue

    match (actual, pattern) with
    | (_, Constant (CharValue '_')) -> Some []
    | (a, Constant v) when a = v -> Some []
    | (a, Variable x) ->
        match lookupValue x with
        | Some (ADTValue (a, b, c)) -> tryMatch actual (Constant(ADTValue(a, b, c)))
        | _ -> Some [ (x, a) ]
    | (ADTValue (name, _, values), Apply (Variable (callName), exprs)) when name = callName ->
        forAll tryMatch values exprs
    | (TupleValue (v1, v2), Tuple (p1, p2)) ->
        match (tryMatch v1 p1, tryMatch v2 p2) with
        | (Some (v1), Some (v2)) -> Some(v1 @ v2)
        | _ -> None
    | (ListValue (valList), List (exprList)) when valList.Length = exprList.Length -> forAll tryMatch valList exprList
    | (ListValue (h :: t), ConcatC (h', t')) -> forAll tryMatch [ h; (ListValue t) ] [ h'; t' ]
    | _, _ -> None
\end{lstlisting}

The variable \texttt{actual} is the value from the \texttt{matchExpression} and \texttt{pattern} the current pattern from the list of patterns, passed in shown on line 4 on listing \ref{interpreter-pattern}, with the \texttt{patternList} from line 8.
Some interesting cases in the \texttt{tryMatch} are for example the \textit{wildcard} pattern: \texttt{(\_, Constant (CharValue'\_'))}, this means that if the pattern is an underscore symbol, it should be matched no matter the actual value of the \texttt{matchExpression}.
It is worth noting that the \texttt{tryMatch} function returns a \textit{Some} with a list of tuples. The tuples are bindings of the pattern to the match value, this makes the value available for further use in the body of the pattern match. When all bindings are collected the correct match-pattern combination is being searched for. If found, it will be added to the environment and the body of the pattern evaluated.

\newpage

\section{Reducer}
The essence of what we are going to do is that we are going to exploit as much of the statically known data in the program, by reducing it into a smaller residual program, which is then compiled into \gls{html} and JavaScript. The program we are reducing is the \gls{mvu} described in the introduction \ref{introduction}. Reducing this program will allow us to optimize the entire diffing cycle, because we can generate JavaScript functions that know exactly what document elements to change, when called. This is all to achieve improved efficiency compared to current \gls{vdom} diffing. However, in order to understand and apply it, partial evaluation needs to be understood.

\subsection{Partial evaluation}
When all inputs to a program are given, an interpreter evaluates the program, and obtains a result. However, this is possible only when all of the inputs are known. 
Partial evaluation is a program optimization technique concerned with specialization and evaluation of programs where only parts of the inputs are known. 
\\\\
Most developers are familiar with specialization from partial application of functions. Partially applying a two-argument function obtains a one argument function where the first value has been \textit{fixed} to the given value. Fixing the variable to a specific value is called specialization. 
\\\\
A partial evaluator is an algorithm which, when given a program and some of its input, will attempt to execute the program as far as possible, and output a reduced \textit{residual} program. When given the remaining inputs the residual program will execute the rest of the program. It works as if the input has been \textit{incorporated} into the original program, as much as possible has been evaluated, and unnecessary branches have been reduced away. Evaluation of the residual program with the remainder of the inputs should yield the same output as evaluation of the original program with all of the inputs. In that sense, it is a specialized version of the original program. Partial evaluation is also known as \textit{program specialization}.
\\\\
The below Figure \ref{specialize-easy} is an example from \cite{Sestoft} showing a two-input program p for computing $x^n$. Partially evaluating the program with the static parameter 5 yields the second specialized program p5.

\begin{lstlisting}[columns=fullflexible, label={specialize-easy}, language=Other, caption={Specialization of a program to compute $x^n$}]
f(n,x) =
  if n = 0 then 1
  else if even(n) then f(n/2,x)^2
  else x * f(n-1,x)

f5(x) = x * ((x^2)^2)
\end{lstlisting}

The reason why this is beneficial is for efficiency gains. Partially evaluating the program with 5 yields the program \texttt{p5} where this expensive computation has already been done. Thus \texttt{p5(x)} is a much faster program than \texttt{p(5)(x)}. It is worth to note that this optimization is possible because n determines the control of the program. If \texttt{x} was the static parameter, it would not be possible to achieve the same optimization. 
\\\\
Figure \ref{specialize-update} shows another example, of the two-input function \texttt{update}, which either increments or decrements the model by one, based on the given \texttt{msg}. Partially evaluating the function with the static parameter \texttt{Increment} yields a new function which always adds 1 to the model it receives. 

\begin{lstlisting}[columns=fullflexible, label={specialize-update}, language=JaLi, caption={Specialization of the update function on increment}]
func update msg model =
  match msg with
  | Increment -> model + 1
  | Decrement -> model - 1
end

func updateIncrement model =
  model + 1
end
\end{lstlisting}

Let's assume that each branch contains some expensive computation. Partially evaluating \texttt{update} with \texttt{Increment} yields the function \texttt{updateIncrement} where this expensive computation has already been done. This is desirable when the function is called with several different second parameters, as it allows to reuse \texttt{updateIncrement} with several different arguments without recomputing the expensive computation. Thus efficiency can be achieved by partially evaluating the \texttt{update} function with each of the inputs \texttt{Decrement} and \texttt{Increment}, and replace all calls to \texttt{update(Increment)} with the specialized function \texttt{updateIncrement}, and replace all calls to \texttt{update(Decrement)} with the reduced function \texttt{updateDecrement}.

\subsection{Approach}
So what is actually going on in partial evaluation, is a combination of evaluation and code generation. We are evaluating all calculations that depend only on the known input. These expressions are called \textit{static}. All expressions that rely on unknown inputs are called \textit{dynamic}. For each dynamic expression we generate code by replacing it with a new expression. This is described by the following techniques. 
%\liv {Currently we are reducing expressions away, e.g. when %reducing Function(..), but maybe we shouldn't?: '\textit{In %general, reduction of the static version (at specialization %time) will produce a value, whereas reduction of the dynamic %version will not change its form, only reduce its %subexpressions}'.)}
There are three main techniques in partial evaluation \cite{Sestoft}: \textit{symbolic computation}, \textit{unfolding function calls}, and \textit{program point specialization}. The two first techniques have been sufficient for this project and will be described here, while the third one will be described in improvements.
\paragraph{Symbolic computation:} This is the process of computing with symbolic values either by rewriting or evaluation. Symbols, in this context expression, represent rewritable terms, while values imply the end of rewritability. We are going to specialize function bodies by reducing it symbolically; we will rewrite dynamic expressions to other expressions (code generation) and evaluate static expression into values.
\paragraph{Unfolding:} This means replacing a function call by a copy of \texttt{f's} body where all parameter variables have been replaced by the corresponding arguments. E.g. if there is somewhere in the program that the partial evaluator finds a call to \texttt{unfold Increment 5} it will replace the call with the constant 6.
\paragraph{Unfolding Strategy:} Unfolding can either be done \textit{on the fly} or in a post phase. We will be using the former. To avoid infinite unfolding, Sestoft \cite{Sestoft} proposes three strategies.
\begin{enumerate}
    \item Avoid infinite unfolding by not unfolding function calls, however this would mean that the function calls would not be unfolded even though the arguments are static, resulting in a minimal specialization.
    \item Only unfold function calls when all parameters are static. This risks only infinite unfolding if the original program had a potential \textit{infinite static loop}, i.e. a loop that does not involve any dynamic tests.
    \item Not to unfold a function call inside a dynamic conditional. As before, this avoids infinite unfolding as long as the original program does not contain a potential infinite static loop itself.
\end{enumerate}

We will be using the latter strategy, however since the control flow in our language is also determined by pattern matching, we will extend the constraint to also apply in dynamic match expressions. 

\paragraph{Offline and online partial evaluation:}
One should note that there are two different processes for partial evaluation: online and offline. Offline divides the partial evaluation into two stages. The first is a preprocessing phase, where expressions are divided in static and dynamic expressions, e.g. by annotating each expression as either dynamic or static, based on the available information. Then during the actual specialization, the decision of whether a certain expression should be reduced or not is based on the precomputed division. Online processing consists of only one phase: the action to take at each expression is decided based on the concrete values computed during specialization. The main advantage of the online process is that it exploits more static information during specialization than the offline process \cite{Sestoft}. The online process will be followed.

%** Sestoft book ** 
%- pre-compute all expressions involving n
%- unfold the recursive calls to function f
%- reduce x*1 to x.

%- Partial evaluation can even be advantageous in a \textit{single run}, since it often happens that partial evaluation p on in1 and then running on in2, is faster than running p on in1 and in2. An analogy is that compilation plus target run time is often faster than interpretation in Lisp.

% Does partial evaluation eliminate the need to write compilers? Yes and no...Pro: ... Contra: the generated target code is in the partial evaluator's output language, typically the language the interpreter is written in. ... 

%** Sestoft book end **

\subsection{Implementation}
Our partial evaluator is going to produce a residual program by running through the program and pre-compute as much as possible, based on static variables and arguments to function calls. Having detailed the expressions of the JaLi language in Chapter 3, here it will be explain how each of them are reduced by the partial evalutor. 
\\\\
\textit{A note on the implementation}: The reduction of the expressions is mostly trivial, however reducing the patterns are quite complex. We believe that this is mainly caused by the abstract syntax behind JaLi, mainly in the different representations of \glspl{adt}. A much simpler implementation of the reducer is very likely, with a better implementation of the abstract syntax. Another notable thing is that this is not an optimal implementation of the reducer, as it may reduce function bodies and other expressions even though no static information is available and thus no reduction will be achieved. For now, this works just fine for the examples demonstrated.
\\\\
The partial evaluator is implemented as the single function \texttt{reduce} as seen in listing \ref{reduce-base}. As main parameters it takes an expression to reduce and a store. The store is a list of variable bindings similar to the environment parameter in the interpreter; it is a list of (string, Expr) tuples, mapping a name to an expression. The last parameter is the \textit{context} which expresses whether it is inside a dynamic conditional. This is to implement the unfolding strategy that does not unfold function calls inside dynamic conditionals. When done, the reduce function outputs an Expr.
\\\\
Reducing a constant returns the constant. Reducing a variable will look up the variable in the store and return the expression. 

\lstinputlisting[columns=fullflexible, label={reduce-base}, language=JaLi, caption=The reduce function]{./code/compiler/base.jali}

Listing \ref{reduce-ctl} shows the reduction of concatenations, tuples and lists, as these cases resemble each other. We reduce each of the sub-expressions. For concatenation and lists we return a list and for tuples we return a tuple with the reduced sub-expressions.

\lstinputlisting[columns=fullflexible, label={reduce-ctl}, language=JaLi, caption=Reducing concatenations tuples and lists operations]{./code/compiler/concat-tuple-list.jali}

As shown in listing \ref{reduce-prim}, a primitive operation is reduced similarly to the previous, by reducing each sub-expression. If both are constant, the interpreter will be used to evaluate the primitive expression and return a constant value. However, even though one of the sub-expressions are not constant, it may still be possible to reduce the expression. E.g. multiplying a variable with 0 will always yield zero, as seen in the match cases in the listing. For the problems that are being showcased, it suffices to consider only these cases. However, the list is not exhaustive, and a complete partial evaluator should consider many more cases in order to be optimal.

\lstinputlisting[columns=fullflexible, label={reduce-prim}, language=JaLi, caption=Reducing primitive operations]{./code/compiler/prim.jali}

Listing \ref{reduce-let} shows reduction of let-bindings. When reducing let-expressions the body will be reduced first. Then the result will be added to the store, bound to the name of the binding variable. Lastly the following expression will be reduced. This effectively removes the let binding expressions. The expression lives in the store and replaces dynamic variables at the places where it is referenced.

\lstinputlisting[columns=fullflexible, label={reduce-let}, language=JaLi, caption=Reducing let bindings]{./code/compiler/let.jali}

Reducing conditionals is shown in listing \ref{reduce-cond}. First, the conditional expression is reduced. If the expression is a constant boolean value, we already know what branch the program will take, and thus we return the reduction of this branch. If the conditional is not a constant, we return a conditional expression with the branches reduced. As according to our unfolding strategy, we change the context to false, expressing that we are inside a dynamic conditional, and thus should not unfold function calls when encountering them.

\lstinputlisting[columns=fullflexible, label={reduce-cond}, language=JaLi, caption=Reducing conditional expressions]{./code/compiler/cond.jali}

Listing \ref{reduce-function} shows the reduction of functions. In order to handle recursive calls, we add the closure to the store before reducing the function body. Then we create a new closure with the reduced body, add it to the original store, and continue by reducing the second expression that follows the function. Here we always return the reduced expression that follows the function. The function exists as a closure in the store, which will be specialized and unfolded at function calls.

\lstinputlisting[columns=fullflexible, label={reduce-function}, language=JaLi, caption=Reducing functions]{./code/compiler/function.jali}

Reducing \gls{adt} definitions happens in listing \ref{reduce-adt} by adding the constructors to the store and continue reducing the following expression. If the constructor has no parameters, it is an \gls{adt} value and otherwise an \gls{adt} constructor. A simpler representation of \gls{adt} constructors would be desirable, as the differences causes complexity in the patterns matching.

\lstinputlisting[columns=fullflexible, label={reduce-adt}, language=JaLi, caption=Reducing \gls{adt} declarations]{./code/compiler/ADT.jali}

Function application in listing \ref{reduce-apply} are reduced by first reducing the expression to apply and the arguments given. The reduced expression may either be a closure, \gls{adt} closure or a constant. When reducing a closure, we check if the context is true, meaning that are not in a dynamic conditional, and thus may unfold function calls. That is, we replace the function application with the specialized function body of the closure. If the context is false, we do not unfold the function body, but rather return and apply the expression. This results in not being able to handle partial function application, which is no problem as this suffices for the examples at hand. If the closure is an \gls{adt} closure, we check if all required arguments are given. If the arguments are all static, we return a constant \gls{adt} value, and otherwise return an \texttt{Apply} to the \gls{adt} closure. 

\lstinputlisting[columns=fullflexible, label={reduce-apply}, language=JaLi, caption={Reducing function application}]{./code/compiler/apply.jali}

The complexity of the reducer lays in the reduction of patterns. The reduction of patterns is given in listing \ref{reduce-pattern}, but which utilizes the \texttt{match1} function given in listing \ref{reduce-util}. The \texttt{match1} will try to match a single pattern with the match expression. When a pattern contains a variable name, it is a binding of an expression to a name, which shall be used in the body of the patterns. Thus, this is the main task of the \texttt{match1} function: Try to match a value with a given pattern and collect all the bindings in the pattern. This is the complex and main part of this reduction. A matching is represented by the type \texttt{ReducedMatch}, which can take three different forms. Either it is a \texttt{NoMatch}, if it is certain that the pattern will never match, even when the dynamic variables of the match expression is known. A \texttt{DynamicMatch} means that the pattern could possibly match, once the dynamic variables are known. The \texttt{StaticMatch} is certain that the pattern will match. Dynamic and static matches carry a list of bindings, which is a list of tuples of name and expression. 
\\\\
The \texttt{matchMany} function is a helper function for when many expressions determines whether a match is static or dynamic or a no match - e.g. in lists. The function matches all expressions in the list, and checks if one of them is a \texttt{NoMatch}. Otherwise it collects all the bindings that result from the matching. If any of them is dynamic, then the entire result is a dynamic match, and otherwise it is static. 
\\\\
The last helper function is \texttt{collect} which simply traverses the pattern and collects all variables as bindings to itself. This is used when the match expression is itself a dynamic variable. Then we cannot match any further in the pattern, but we still need to collect the rest of the bindings from the pattern in order to reduce the body.
\\\\
Returning back to the reduce function in listing \ref{reduce-pattern}. First the match expression is reduced, and the result is either a constant or it is dynamic. In either case we need to match the expression with the given patterns and reduce the body of the pattern. If the match expression is constant, we know that one of the patterns must match statically. Thus, we just need to find the first static match. The bindings are then added to the store, before reducing the body of the pattern and returning the reduced expression.
\\\\
If the match expression is dynamic we need to do the same thing, we check all patterns and choose the ones that matches. If the pattern is a match, the bindings are added to the store, and the body is reduced. The output is a tuple together with a boolean flag communicating whether the binding was static or dynamic. If the first match in the list is static, we know that this will always match and thus we can simply return the body of that pattern. Otherwise we return a pattern expression. 
\\\\
An important thing to note is that the patterns, like conditionals, control the flow of the program. Therefore, according to our unfolding strategy, the context must be switched to false inside dynamic pattern matching, to avoid infinite unfolding.

\lstinputlisting[columns=fullflexible, label={reduce-pattern}, language=JaLi, caption=Reducing pattern match expressions]{./code/compiler/pattern.jali}

\lstinputlisting[columns=fullflexible, label={reduce-util}, language=JaLi, caption=Function for matching an expression with a pattern expression]{./code/compiler/util.jali}

\subsection{Discussion}
\subsubsection{Complexity} As can be seen in listing \ref{reduce-util}, there are many cases which need to be considered. The complexity is mainly caused by the different representations of \glspl{adt} in the abstract syntax. These are either represented as \textit{ADTValues}, \textit{ADTClosures} or function applications \textit{Apply}. More complexity is added from the different representation of lists, in the form of Lists and ConcatC. This is unfortunate since the reducer has to be clever and extremely precise, in order to work correctly and optimally. Unfortunately this complexity has prevented us from being able to reduce the complex views containing several nested structures. It has however been possible to reduce on smaller examples, which is enough to showcase the point we want to make.

\subsubsection{Program point specialization}  Another improvement is that of \textit{program point specialization}, which as mentioned is one of the three main techniques in partial evaluation that we do not exploit. In a functional language, program points are the names of the functions. Essentially program point specialization is to define and memorize specialized functions. This was exemplified in \ref{specialize-update} where we defined a new function \texttt{updateIncrement}, which can then replace all expressions in the program where \texttt{update Increment model:dyn} is called (here :dyn denotes that the model is unknown, i.e. dynamic). Thus, one function may appear many places in the program in a specialized version with a specialized name, and program point specialization is a combination of defining and folding functions.

The technique would be implemented as follows. Whenever a function call is reduced, we memorize the name and arguments. When the same function call is encountered in a different place in the program (e.g. \texttt{update Increment model:dyn} may be called different places in the program) we lookup if the reduction has been computed previously. If so, we can use a copy of that. If not, the function has not been called with that combination of static and dynamic arguments before, and thus we reduce it now and memorize the reduction.

Currently we are not utilizing program point specialization in the reducer, we simply reduce every time we encounter a function call (and we are not within a dynamic if-branch). This means that we may specialize the same function body with the same arguments many times. Clearly memorizing the reduction would be beneficial in terms of efficiency. 

\newpage

\section{Model-view-update architecture in JaLi} \label{Model-view-update-architecture-in-JaLi}
% Chapter 4
% Result: JaLi implementation of Elm model
%   Give example: button inc/dec
%   Show what it looks like in code
%   Give all examples, what it suppose to do
%   State what everything needs to be called if not reduced
%     The intended execution model (and that is why we want to reduce: so we
%     don’t need to go through all this)
%   runtime, how much in the language itself (almost everything)
%   Compile vs. having JaLi code generate JavaScript

% \jan{
% \begin{enumerate}
%     \item No JavaScript compiler -> no state in frontend
%     \item Small snippets of JavaScript that do the change (like button example)
%     \item 
% \end{enumerate}
% }

With the JaLi language in place, this chapter will show how the \gls{mvu} architecture can be implemented using JaLi. The button example will be continued and his somewhat influenced from the Elm button example.
The program will have a model, that holds the current state of the program. An update function defining how to handle each possible action, like \textit{Increment} and \textit{Decrement}. The last function \textit{view} is responsible of taking the model's value and transforming it to \gls{html} for the browser to render.
An \gls{adt} named \texttt{Msg} will hold the different actions that are possible. The \gls{html} structure can be represented by a \texttt{Node} type, which was explained in chapter \ref{jali}.
The \texttt{view} function takes the model and places into the \texttt{Tag} structure that is going to be the \gls{html} in the end, but returns here just a structure of type \texttt{Node}. This can then be passed to \texttt{viewToHtml} which translates the a value of type \texttt{Node} into a string containing valid \gls{html}.
This is good enough to prepare one static view that can be rendered in the browser.
To make it dynamic it needs a \textit{handler}, that attaches to all the places where a user can interact with the website. For the button program it would make most sense to attach an on-click handler, that is triggered when a button is being clicked. In order to make this work in the browser, all actions need to be written in JavaScript, as the browser does not execute any other programming language on the client. This is no problem, but requires a JavaScript compiler, which is unfortunately missing, but can be simulated to a limited degree.

\begin{figure}
    \centering
\begin{verbatim}
type Node
type Msg
update: Msg -> Model -> Model
view: Model -> Node 

type Differ
diff: Node -> Node -> Differ
patch: Differ -> Node -> Node

patchToJs: Differ -> String
viewToHtml: Node -> String
\end{verbatim}
    \caption{Pseudo code to show functions, types, \glspl{adt}}
    \label{fig:mvu_types}
\end{figure}

For the button example the actions \textit{Increment} and \textit{Decrement} alter the displayed value, by either incrementing its value or decrementing by one.
To make this as generic as possible and adapt previous ideas like \gls{vdom} comparing, two functions will be implemented: \texttt{diff} and \texttt{patch}. These are also the functions that need to be translated to JavaScript and executed -- with the update function -- on every action.
The \texttt{diff} function detects changes between two views and the \texttt{patch} function takes these changes and patches them in the view that is being shown.

Constructing two different views by calling the \texttt{view} function with 1 and 2 would, when given to the \texttt{diff} function, return the \texttt{Differ} from listing \ref{non-reduced-differ}.

\begin{lstlisting}[columns=fullflexible, label={two_views}, language=JaLi, caption=Two views to compare]
view1 = Tag ('div') [] [
    Tag ('button') [] [Text ('Increment')],
    Tag ('p') [] [Text ((*@\textcolor{blue}{1}@*))],
    Tag ('button') [] [Text ('Decrement')],
];
view2 = Tag ('div') [] [
    Tag ('button') [] [Text ('Increment')],
    Tag ('p') [] [Text ((*@\textcolor{blue}{2}@*))],
    Tag ('button') [] [Text ('Decrement')],
];
\end{lstlisting}

Listing \ref{diff_function} shows the definition for the \texttt{diff} function. It receives two views and steps through it comparing always the first element in the list of nodes, if both \texttt{Tags} are the same it continues comparing, by using the helper function \texttt{fold} to iterate over the children nodes on the compared tags.
If it encounters a difference either on the \texttt{Tag} or on the \texttt{Text} it returns a \texttt{Change} constructor with the value from view2, as this is the newer view.
\texttt{fold} basically just takes the first element out of the list and calls \texttt{diff} on it again, if that call returns a \texttt{Change} it indicates that there is a change by returning a \texttt{Path} to show the way to get to that change.
This function at this stage only detects the first difference on two views and then terminates. This is for the trivial button example sufficient, but should be extended for real use.

\lstinputlisting[columns=fullflexible, label={diff_function}, language=JaLi, caption=Diff function detecting changes on two views]{./code/diff.jali}

\begin{lstlisting}[columns=fullflexible, label={non-reduced-differ}, language=Other, caption=Differ path detecting change on \texttt{Text} node]
$ Constant
(ADTValue
   ("Path","Differ",
    [IntegerValue (*@\textcolor{blue}{1}@*);
     ADTValue
       ("Path","Differ",
        [IntegerValue (*@\textcolor{blue}{0}@*);
         ADTValue
           ("Change","Differ",
             [ADTValue ("Text","Node",[IntegerValue (*@\textcolor{blue}{2}@*)])])
        ])]))
\end{lstlisting}

In order to apply this detected change the \texttt{patch} function takes the old view, the changes and returns a new view.

\lstinputlisting[columns=fullflexible, label={patch_function}, language=JaLi, caption=Patch function applying changes on old view]{./code/patch.jali}

It does this by iterating over the \texttt{Differ} \gls{adt} held in \texttt{changes}, which is a recursive path structure that has a \texttt{Path} and \texttt{Change} constructor. When a \texttt{Change} is encountered it immediately returns the value from it, otherwise it will take the index from the \texttt{Path} and find the corresponding \texttt{Tag} node in the view and follow it recursively by calling patch again.
It uses two helper functions: \texttt{mapi} and the inner function called \texttt{f}. \texttt{mapi} iterates through a list a applies the given function on each element it passes while maintaining an index, that is passed to the function. This allows the tracking of elements in the view and the index from the \texttt{Path}.
Ultimately \texttt{patch} will then return an updated \texttt{Node} structure that looks like \texttt{view2} from listing \ref{two_views}.

This whole process is then finished, but by now only shown on two views, that means the second view, was already constructed, triggered by something. This shall be explained now and how that might look in JavaScript after all.

\jan{write about the JavaScript that is being generated. What the handler actually triggers}


% \liv{should we just get rid of 'patch'?}
% \jan{no, as it is useful to show what the JaLi can do}
% The view function, takes a model and returns a string that represents the structure in form of valid \gls{html}. The update function takes an action and a model and updates the model accordingly. To demonstrate how this works, we construct a simple example consisting of two buttons: \textit{Increment} and \textit{Decrement} and a paragraph element displaying a counter, that can be altered by using the buttons respectively. The corresponding JaLi program can be found in figure \ref{JaliMeetsElm}. 

% \lstinputlisting[label=JaliMeetsElm, language=Jali, caption=Button example in JaLi]{./code/button.jali}

% To store the state of the counter it has a simple model which is just an integer set to 0. 

% The \texttt{view} function describes the view, in terms of the \gls{adt} \texttt{Node}. To illustrate how this data type can be used to represent an \gls{html} page with a counter an to buttons, see the \texttt{view} function in the code listing \ref{JaliMeetsElm}. A \texttt{Node}, can either be a \texttt{Tag}, representing \gls{html} elements like \texttt{<div></div>} or \texttt{<p></p>}, or a \texttt{Text} node, which represents the plain text that can be written within \gls{html} elements. A \texttt{Tag} takes three arguments: a string to denote the type of the tag (\textit{button}), a list of key-value pairs that are called attributes (\textit{value=42}), and a list of child nodes, which are the nested tags in this element. A \texttt{Text} node is a leaf node and can only contain text.
% In this example, the \texttt{view} function takes an integer as argument and returns the view with the model inserted as the counter: when the top button is pressed, \texttt{update Increment} is called, and when the lower button is pressed, \texttt{update Decrement} is called. 

% By giving the view function the initial model 0 we get an ADT of value of 'Tag' corresponding to an \gls{html} page with two buttons and a counter set to 0.

% \liv{ How to proceed? }
% \jan{explain the other functions and show example outputs}
% \liv{should we show the generated adt?} 
% \jan{yes}
% % we use the evaluater on the Differ (Diff (view init) (view (update Increment) init) -> and then translate that to javascript
% \jan{write about diffing and what happens and pretend JavaScript compiler exists, show outputs}

\newpage

\section{Reducing the model-view-update architecture}
% Chapter 5
% Reducing the Elm model
%   Apply reducer, what do we get
%   How far away is the plain JavaScript button implementation to ours from JaLi
%   Yeah
%     javascript2 = reduce ( patchToJs (diff (view model) (view (update Decrement model))))
  
% The patchToJS function generates JavaScript that makes changes during runtime to the model.
% __________________________________________________________________________________
This chapter focuses on the reduction of the model-view-update architecture and its result when using the JaLi compiler with partial evaluation implemented.
The previously shown example of two buttons with a counter text field shall be continued here.

To revisit the desired outcome of having two small JavaScript snippets that sit on the buttons that apply the change directly onto the \gls{dom} node, it needs to be figured out now, how to achieve this by reducing the model-view-update architecture.

When an update on the \gls{dom} is happening \textit{diffing} (see chapter \ref{Model-view-update-architecture-in-JaLi}) needs to take place.
This diffing however is expensive, because it needs to iterate through both views and compare their nodes, until it has calculated the complete \textit{Differ} path, knowing where all changes are that need patching. For this reason the diffing should only happen when absolutely necessary or even better happen during compile time and reduced to optimized functions, that can be directly used.

Continuing with the button example from the previous chapters, it is intuitive to see that the two increment and decrement buttons will always only change the model by either incrementing or decrementing its value by one. It is also only being rendered by one single \gls{dom} node.

\begin{figure}[H]
    \centering
    \begin{verbatim}
    func view model =
      Tag ('div') ([]) ([
        Tag ('button') ([
            (Click, (onClick (Increment)))]) ([Text ('Increment')
        ]),
        Tag ('p') ([]) ([
          Text (model)
        ]),
        Tag ('button') (
            [(Click, (onClick (Decrement)))]) ([Text ('Decrement')
        ]),
      ])
    end
    \end{verbatim}
    \caption{ADT of button example}
    \label{fig:button-view-function}
\end{figure}

The only problem when wanting to reduce, is that during compile time when the reducer is applied the model cannot be known. Using an initial value 0 on the model, would cause the problem that the reducer would generate two functions that either add or subtract 1 to 0. Therefor two functions need to be generated where the model is treated as dynamic:

\begin{verbatim}
updateIncrement: update Increment dynamic:model
updateDecrement: update Decrement dynamic:model
\end{verbatim}

This also means, that in the end our run-time needs to hold some state for the model, which can then be passed to the partially evaluated \texttt{updateIncrement} and \texttt{updateDecrement} functions.
The state would be generated by a JavaScript compiler, that would compile the whole JaLi program into JavaScript, this JavaScript compiler is not implemented. 
The update on the model is not enough though, since the \gls{dom} needs to be patched as well. The \textit{onClick} handler attached to the button essentially calls a function: \textit{updateAndPatch}.

\begin{figure}[H]
    \centering
    \begin{verbatim}
      func updateAndPatch action = 
        oldModel = readGlobal ("model");
        newModel = update (action) (oldModel);
        eval (patchToJs (diff (view newModel) (view (oldModel))))
        setGlobal ("model") (newModel)
      end
    \end{verbatim}
    \caption{Function called by onClick handler}
    \label{fig:updateAndPatch}
\end{figure}

Figure \ref{fig:updateAndPatch} shows four operations performed when called.
\begin{enumerate}
    \item It reads a global variable that is called \textit{model} and assigns it to \textit{oldModel}
    \item Generates a new model based on the \textit{Action} passed and the \textit{oldModel}
    \item Evaluates the JavaScript string returned from the patch on the old view and new view
    \item Set the global variable to the value of \textit{newModel}
\end{enumerate}

The two parts that get reduced here and are crucial for performance gain is the \textit{update} and the \textit{diff} functions.
These two function calls would be reduced as there are two different actions:

\begin{figure}[H]
    \centering
    \begin{verbatim}
        diff (view dynamic:model) (view (update Increment dynamic:model))
        diff (view dynamic:model) (view (update Decrement dynamic:model))
    \end{verbatim}
    \caption{Increment and Decrement diff calls}
    \label{fig:diff-inc-dec}
\end{figure}

The reduction of these two calls will yield a a partially evaluated function, that will always return the same \textit{Differ} path, and only the actual \textit{Change} constructor from the \textit{Differ} data type, will be different.
This means that for all actions that update the model and thus change the view, will be calculated and reduced during compiling to the point that only the value of the to be inserted node is missing, which means no \gls{dom} compare/diffing needs to be done during run-time.

All other parts in \ref{fig:updateAndPatch} will be treated by the reducer as dynamic -- the setting of the global model -- and hence not reduced further.



\jan{show reduced diff function}
\jan{show the parts in fig 3}

The JavaScript compiler would then generate JavaScript code from the whole JaLi program and generate a run-time that would have access to a \textit{model} that could be passed to the partially applied JavaScript function generated by the compiler based on the reduction of the whole program

\jan{TODO: here show JavaScript snippets that make the change}

\jan{TODO: 
- JavaScript snippets
- Reduction of Differ
- Reduction of patchToJs
}

\begin{verbatim}
document.body.children[0].children[1].children[0].innerHTML = 1;
\end{verbatim}

\jan{backlog: write JaLi function eval that calls JavaScript eval}

% \begin{figure}
%     \centering
%     \begin{verbatim}
%     func incrementDynamicPatchToJs dynamic_model = 
%         view1 = view (dynamic_model);
%         view2 = view (update Increment dynamic_model);
%         patchToJs view1 (diff view1 view2)
%     end
%     \end{verbatim}
%     \caption{Patch function with dynamic model}
%     \label{fig:dynamicPatchToJs}
% \end{figure}

% The \textit{dynamicPatchToJs} function from Fig. \ref{fig:dynamicPatchToJs}. shows how it generates two views on a dynamic model, where one view is being updated by an \textit{Increment}.

% The partially evaluated \textit{diff} function is not yet enough though, because it still needs to be used to patch the view.
% As previously stated that the patching cannot be done during compiling, due to the lack of information regarding the dynamic variable model, the \textit{patch} function needs to generate JavaScript that can be executed during run-time, where the model is known in form of a state variable.
\newpage

\section{Conclusion}
Partial evaluation is a valuable method of gaining major performance gains in a straightforward way, that starts at the compiler level, whereas most of the time optimizations are searched in the coding style of a program or in using clever architectural patterns.
With the complete move of comparing tree-structures in the run-time to the compile time and there doing it only once for each possible action, programs can be made significantly more efficient. 
This was shown by developing a new functional language called JaLi, which implements a compiler that has to limited degree partial evaluation, symbolic execution and constant folding as optimizations.
The JaLi language was used to introduce the \gls{mvu} pattern and show its drawbacks when performing diffing and patching. A reduced example, with \glspl{adt} representing a simple \gls{html} structure was then taken and reduced by the compiler. Doing the diffing and patching during compilation for a given action, generating a JavaScript snippet that could be used to change the \gls{dom} dynamically.

Reducing the \gls{mvu} implementation of a JaLi written web program in chapter \ref{reducing-mvu} efficiently showed how partial evaluation may achieve the complete circumvention of the \gls{mvu} cycle. This happens by precomputing all the diffing at compile time. This would enable considerable performance gains for web frameworks.


\section{--Old stuff--}
We will attempt to optimise away (a) the generation of intermediary \gls{dom} representation and (b) diffing of intermediary and actual \gls{dom} by using techniques from partial evaluation.
\\\\
Either we use an existing language and write a new compiler to do our planned compiler optimisation or we create own language that is small enough to demonstrate example and has compiler optimisation. We have chosen the latter. 
\\\\
Having implemented our own language, we now implement the Elm-like web-framework that can be used generate and update an \gls{html} website. As in an Elm-like model-view-update, we have two explicitly states functions 'update' and 'view':

\begin{verbatim}
    update: Action -> Model -> Model
    view: Model -> View 
\end{verbatim}

The view function, takes a model and outputs an \gls{html} like structure, describing how the view should look like. The update function takes an action and a model and updates the model accordingly. We then have a diff function, which takes two views, and returns a data structure that describes the path to all differences between the two views.

\begin{verbatim}
    diff: View -> View -> DOM-Update-Actions
\end{verbatim}

Assume that we have a current view based on the current model m. When a user event triggers some action a on the model m, the functions are applied as follows:

\begin{verbatim}
    diff current_view (view (update a m))
\end{verbatim}

We update the model m with a, and then call view to get the view of the new model. Then we diff the old view with the new view, to get the changes that needs to be applied to the current view. 
\\\\
Now all we need is a patch function, which takes the initial view and the returned structure from the differ function and applies the changes onto the passed view, returning a new view with all fields updated.

\begin{verbatim}
    patch: View -> DOM-Update-Actions -> View
\end{verbatim}

To demonstrate that this works, we do it by example: Take as example a website with a single counter on the screen, and two buttons; one that increments and one that decrements the counter. The corresponding Jali program can be found below. To store the state of the counter it has a simple model which is just an integer set to 0. It has two actions: Increment and Decrement. The update function either increment or decrements the model, depending on the given action. Lastly we have the view function, which describes how to turn a given model into an \gls{html} like structure. The view is described as the \gls{adt} type 'Node' in the top, which can either be a 'Text' node containing a string, or a 'Tag' node containing a name, a list of attributes as tuples and a list of children.

\lstinputlisting[language=Jali, caption=Jali example]{./code/button.jali}

Running the Jali program through the interpreter will generate our initial view represented as an \gls{adt} value of 'Tag'. This \gls{adt} can now be translated into \gls{html}. The function that translates the \gls{adt} to \gls{html} and the \gls{html} output can be found in Appendix ??. Figure \ref{fig:buttons_in_browser} shows the generated \gls{html} displayed in the browser.   

\lstinputlisting[language=Jali, caption=View example]{./code/button-output.fs}

Add html figure here:
\begin{figure}
    \centering
    % \includegraphics{}
    \caption{Todo: caption}
    \label{fig:buttons_in_browser}
\end{figure}

\liv{How to display the diff and patch function? }

However, pressing these buttons will not do anything without any JavaScript code. That is, from our model, we also need to generate the JavaScript that knows what should happen when we press the button. This is where partial evaluation will help us. 

We have a diff function and a patch function which will figure out the change we need to make, and patch the view according to that change. This is applied as follow: 
\begin{verbatim}
    patch (diff current_view (view (update a m)))
\end{verbatim}

\liv{Explain about the reducer?}
\liv{Explain about patchtojs?}
We take the reducer, we apply it as follows:

\begin{verbatim}
    reduce (patchtojs (diff current_view (view (update a m))))
\end{verbatim}





\textit{Notes: Our Jali program works very similar to ELM. We have 4 functions: update, view, diff and patch. \\
- First, we need a view function, which describes how to turn our model into \gls{html}\\
- We need the update function, which takes an action and a model and updates the model accordingly. \\
- The differ function takes two views and will return a data structure that describes the path to all differences between the two views.\\
- The patch function takes the initial view and the returned structure from the differ function and applies the changes onto the passed view, returning a new view with all fields updated.}]





\paragraph{From reducing elm architecture}

So having implemented ... we apply it to the original button example. That is, we implement the view and update functions in jali to construct a view with a button that increments a counter, and with the initial model being set to zero.
1) We generate the html of the initial view/model using our viewToHtml function, and
2) each place that calls the update function, we can generate the corresponding JavaScript that will make this update to the model at runtime, and then we attach the JavaScript to the button. Lets use our very simple button example; we have a model = 0 and a view with a single increment button along with a 'div' displaying the model. We generate the html using our htmlToView function, and the JavaScript for the increment button using our patchToJs method: The generated JavaScript will then simply set the inner html of the counter to 1.

\begin{verbatim}
    html = viewToHtml (view model)
    inc_javascript = patchToJs  (view model) (diff (view model) (view (update Increment model)))
\end{verbatim}

The html and javascript output can be seen in \ref{???}. To see that it works, save it to an html file, open it in the browser and press the button.

\textbf{Insert html and javascript output}

Now the JavaScript here simply sets the inner html to 1, so it really only works one time: it simply sets the counter to a fixed number, calculated from the model that was used as argument when generating the JavaScript. This is not exactly what we want: what we want is a JavaScript function that increments the value based on the current value of the model. 


So we need a JavaScript function that takes one argument, the model, and then sets the counter to the given model incremented by 1.\\
HERE WE USE THE REDUCER -> GENERATES THE CORRESPONDING JALI FUNCTION -> AND THEN THE COMPILER TO CONVERT IT TO JAVASCRIPT. SEE BELOW\\
(* By reducing 'dynamicPatchToJs' in the below example, we obtain a Jali function that needs one last argument, and outputs JavaScript that increments the model. Compiling this output to JavaScript will generate a corresponding JavaScript function, that takes one argument, the model, and outputs the JavaScript that increments the given model. We attach the function to the Increment button, and now every time it is pressed, it will call this JavaScript function with the current model. *)

\newpage

% \subsection{A Subsection Sample}
% Please note that the first paragraph of a section or subsection is not indented. The first paragraph that follows a table, figure, equation etc. does not need an indent, either.

% Subsequent paragraphs, however, are indented.

% \subsubsection{Sample Heading (Third Level)} 
% Only two levels of headings should be numbered. Lower level headings remain unnumbered; they are formatted as run-in headings.

% \paragraph{Sample Heading}
% The contribution should contain no more than four levels of headings. Table~\ref{tab1} gives a summary of all heading levels.

% \begin{table}
% \caption{Table captions should be placed above the tables.}\label{tab1}
% \begin{tabular}{|l|l|l|}
% \hline
% Heading level &  Example & Font size and style\\
% \hline
% Title (centered) &  {\Large\bfseries Lecture Notes} & 14 point, bold\\
% 1st-level heading &  {\large\bfseries 1 Introduction} & 12 point, bold\\
% 2nd-level heading & {\bfseries 2.1 Printing Area} & 10 point, bold\\
% 3rd-level heading & {\bfseries Run-in Heading in Bold.} Text follows & 10 point, bold\\
% 4th-level heading & {\itshape Lowest Level Heading.} Text follows & 10 point, italic\\
% \hline
% \end{tabular}
% \end{table}


% Please try to avoid rasterized images for line-art diagrams and schemas. Whenever possible, use vector graphics instead (see Fig.~\ref{fig1}).

% \begin{figure}
% \includegraphics[width=\textwidth]{fig1.eps}
% \caption{A figure caption is always placed % below the illustration.
% Please note that short captions are centered, while long ones are justified by the macro package automatically.} \label{fig1}
% \end{figure}

%
% the environments 'definition', 'lemma', 'proposition', 'corollary',
% 'remark', and 'example' are defined in the LLNCS documentclass as well.
%

% For citations of references, we prefer the use of square brackets and consecutive numbers. Citations using labels or the author/year convention are also acceptable. The following bibliography provides a sample reference list with entries for journal articles~\cite{ref_article1}, an LNCS chapter~\cite{ref_lncs1}, a book~\cite{ref_book1}, proceedings without editors~\cite{ref_proc1}, and a homepage~\cite{ref_url1}. Multiple citations are grouped
% \cite{ref_article1,ref_lncs1,ref_book1},
% \cite{ref_article1,ref_book1,ref_proc1,ref_url1}.
%
% ---- Bibliography ----
%
% BibTeX users should specify bibliography style 'splncs04'.
% References will then be sorted and formatted in the correct style.
\bibliographystyle{splncs04}
\begin{thebibliography}{8}
\bibitem{ref_article1}
Author, F.: Article title. Journal \textbf{2}(5), 99--110 (2016)

% \bibitem{ref_book1}
% Author, F., Author, S., Author, T.: Book title. 2nd edn. Publisher,
% Location (1999)

% \bibitem{ref_proc1}
% Author, A.-B.: Contribution title. In: 9th International Proceedings
% on Proceedings, pp. 1--2. Publisher, Location (2010)

% \bibitem{ref_url1}
% LNCS Homepage, \url{http://www.springer.com/lncs}. Last accessed 4
% Oct 2017
\end{thebibliography}
\end{document}