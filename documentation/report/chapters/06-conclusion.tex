Partial evaluation is a valuable method of gaining major performance gains in a straightforward way, that starts at the compiler level, whereas most of the time optimizations are searched in the coding style of a program or in using clever architectural patterns.
With the complete move of comparing tree-structures in the run-time to the compile time and there doing it only once for each possible action, programs can be made significantly more efficient. 
This was shown by developing a new functional language called JaLi, which implements a compiler that has to limited degree partial evaluation, symbolic execution and constant folding as optimizations.
The JaLi language was used to introduce the \gls{mvu} pattern and show its drawbacks when performing diffing and patching. A reduced example, with \glspl{adt} representing a simple \gls{html} structure was then taken and reduced by the compiler. Doing the diffing and patching during compilation for a given action, generating a JavaScript snippet that could be used to change the \gls{dom} dynamically.

Reducing the \gls{mvu} implementation of a JaLi written web program in chapter \ref{reducing-mvu} efficiently showed how partial evaluation may achieve the complete circumvention of the \gls{mvu} cycle. This happens by precomputing all the diffing at compile time. This would enable considerable performance gains for web frameworks.
